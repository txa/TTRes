\documentclass[twocolumn,a4paper,11pt,ari]{article}

\usepackage{amssymb}
\usepackage{amsmath}
\usepackage{alltt}
\usepackage{stmaryrd}
%\usepackage{a4wide}
\usepackage{times}
\usepackage{setspace}
\usepackage[a4paper,left=2cm,right=2cm,top=2cm]{geometry}
%\usepackage{fancyheadings}
 
%\usepackage[PostScript=dvips]{diagrams}
%\input epsf
%\input{prooftree}
%\renewcommand{\printlandscape}{\special{landscape}}
\usepackage{pstricks,pst-node,pst-text,pst-tree,pst-3d}
\usepackage{graphicx}
\usepackage{lscape}
\usepackage{pst-grad}
\usepackage{pst-xkey}
\usepackage{multido}

\newcommand{\HLine}{\rule{\textwidth}{1pt}}
\newcommand{\Head}[1]{{\bf #1}\vspace{-2ex}\\\HLine}
\newcommand{\HeadC}[1]
  {{\begin{center}{\bf #1}\end{center}}\vspace{-2ex}\HLine}
\newcommand{\HeadB}[1]{\Head{\Blue{#1}}}
\newcommand{\Nn}{\mathbb{N}}


 \newcommand{\sem}[1]{[\![{#1} ]\!]}
 \newcommand{\imp}{\Rightarrow}
\newcommand{\subn}[1]{\langle\mathtt{new}_{\mathit{C}}/{#1}\rangle}
\newcommand{\triple}[3]{\{{#1}\}\ {#2}\ \{{#3}\}}
\newcommand{\bt}{\ensuremath{\cal B}}
\newcommand{\ct}{\ensuremath{\cal C}}
\newcommand{\defval}{\text{def}}
\newcommand{\variable}[1]{{\frenchspacing\ensuremath{\textit{#1}}}}
\newcommand{\syntax}[1]{\texttt{\def\{{\char123}\def\}{\char125}#1}}
\newcommand{\structrule}[2]{\dfrac{\strut#1}{\strut#2}}
\newcommand{\partialto}{\rightharpoonup}
\newcommand{\pll}{\mathrel{|}}
\newcommand{\trueval}{\mathit{true}}
\newcommand{\falseval}{\mathit{false}}
\newcommand{\nullval}{\mathit{null}}
\newcommand{\selectop}[3]{\{ {#1}\in {#2}\ |\ {#3}\}}
\newcommand{\setopsop}[3]{{#2} \, {#1}\, {#3}}
\newcommand{\forallop}[3]{\forall {#2}\in {#1}.\, {#3}}
\newcommand{\includesop}[2]{{#2}\in {#1}}

 
%\title{Separation Logics for High-Level Languages with\\ Code Pointers\\ {\small Case for Support}}
%\title{Local and Modular Reasoning  about  Code as First Class Type\\ \emph{---~Case
 % for Support~---}}
%\title{Reasoning about Stored Commands:\\ From Function Pointers to Self-modifying Programs
%\\ \emph{---~Case  for Support~---}}
  
  \title{Type Theory with hidden monotonic Resources\\ \emph{---~Case
  for Support~---}}


 %\title{Verifying Programs with Function Pointers and Self-modifying Features\\ \emph{---~Case
  %for Support~---}}
 

\author{Thorsten Altenkirch \qquad Bernhard Reus\\
University of Nottingham\qquad University of Sussex}

\date{}

\begin{document}
\spacing{.87}
\maketitle

\section*{Previous Research Track Record}

up to 2 pages need to agree the format, rest should be easy. What about references?

\subsection*{Thorsten Altenkirch}
\sloppy Thorsten Altenkirch is a Reader at the University of
Nottingham and co-chair of the Functional Programming Laboratory
currently comprising 4 members of staff and 13 PhD students. His main
research interests are Type Theory, Functional Programming,
Categorical Methods and Quantum Computing and he has published over 50
papers in this areas which are frequently cited (h-index $\geq$
24). He has been Principal Investigator on four EPSRC projects,
including \textit{Observational Equality For Dependently Typed
  Programming} (EPSRC grant EP/C512022/1, \pounds 242,198), and
\emph{Reusability and Dependent Types} (EP/G034109/1, \pounds 244,671)
and co-investigator on two more.  He has participated in a number of
EU projects, in particular in the Coordination Action TYPES and its
predecessors. He organized the annual TYPES meeting in 2007 and
specialized workshops on Dependently Typed Programming, in 2004
(Dagstuhl seminar 04381), 2008 in Nottingham and 2010 in Edinburgh.
Altenkirch is editor of Fundamentae Informatica since 2007 and was PC
member of FLOPS12, MFPS12, MSFP12, QPL11, TLCA11, FICS10, ITP10,
WGP10, TPHOLs09 and CIE08. He was twice (in 2007 and 1012) invited Professor at
Universit\'{e} Denis Diderot for one month each and in Spring 2013 he
is a visiting researcher at the Institute for Advanced Study In
Princeton for the whole term to work on Homotopy Type Theory.

Especially relevant for the current proposal is the joint work of
Altenkirch and Swierstra on \emph{functional semantics of effects}
\cite{alti:beast,alti:tfp08,swierstra:phd} which showed how effects
can be integrated in Type Theory without giving up the appealing
simplicity of the type-theoretic approach. His recent work on relative
monads \cite{alti:fossacs10} (which is inspired by earlier joint work
with Reus \cite{alti:csl99}) demonstrates how categorical concepts can
be successfully exploited to structure type-theoretic developments.


\subsection*{Bernhard Reus} 
Dr Bernhard Reus is a Senior Lecturer in Computer Science at the University of Sussex with special interest in program semantics and  logics. Reus received a distinction (``summa-cum-laude'') for his  PhD in Computer Science from the Ludwig-Maximilians-Universit\"at in Munich in 1996, got a Lectureship at Sussex University in 2000 and was promoted to Senior Lecturer in 2006. He is a member of the Foundations of Software Systems group in the Department of Informatics at Sussex. His recent research contributions have been in the area of program verification and semantics for  programming languages and logics, in particular languages with stored procedures. 
He led a small team to build a verification tool for C-like programs that supports (semi-)automatic reasoning about stored procedures (function pointers). In co-operation with colleagues in Saarbr\"ucken, Copenhagen, Oxford and Paris he developed new models for the elegant reasoning about recursive predicates indexed by recursively defined worlds. The models developed are presheaf-models with extra structure which is relevant to the proposed project.
  
Reus has been working in the area of  \emph{Applied Semantics}  for most of his research career. He co-authored a frequently cited new denotational  semantics for the call-by-name  $\lambda$-calculus with control operators and a derivation of an interpreter for this functional language  (Krivine's machine)   \cite{RS7} and was involved in the development of
  a  structural operational semantics of multi-threaded Java  \cite{R2} and  a  OCL like Hoare-logic for a sequential Java-like language \cite{R12}.
 
    He has also made contributions to foundations, in particular domain theory and type theory.
 One of the main results of the PI was the \emph{logical} development of  \emph{Synthetic Domain Theory}  which is a variation of Domain Theory, suggested by Dana Scott, in which  domains can be treated    simply as sets. % and functions are automatically continuous.
His distinguished  PhD thesis  \cite{R3}  laid out the formal development of  a flavour of Synthetic Domain Theory in a type-theoretical setting based on   few axioms only. Reus  verified  all theorems  \cite{R5}  using  the  proof-checker \emph{Lego} which implements a specific type theory (ECC). \emph{Lego} was  somehow a precursor of the  popular  type theoretical systems \emph{Coq} and \emph{Agda} that will be used in the proposed project.  
In order to ensure  the consistency of this formalization, a realizability model was defined \cite{R4}.    Later this work has been generalised to allow for more models in \cite{RS8} which in turn  has been generalised further by other researchers.

The two PIs have  collaborated once before in the area of type theory, developing 
  an elegant definition of $\lambda$-terms using a generalized form of polymorphism \cite{R15}. 
  At the time they also  co-organised and co-chaired   the TYPES workshop in  Kloster Irsee near Munich.


\textbf{Grants}
Reus was the sole PI on the recent EPSRC project     \emph{From Reasoning Principles for Function Pointers To Logics for Self-Configuring Programs}
      (EP/G003173/1, \pounds 390,871) 
      which investigated models and program logics for higher-order store, i.e.\  theories and tools that allow one to  prove correctness of programs  that use code pointers.   
      One application  of the developed semantics is the full  soundness proof of Pottier's \emph{capability types} \cite{R14}.
       Another outcome is a (semi-)automatic theorem prover, Crowfoot \cite{R3}, that allows one to write C-like programs with stored procedures (function pointers), specify them using pre- and postconditions and automatically prove that the procedures meet the specification using just some annotations.
   
 Reus was also the sole PI on the EPSRC project \emph{Programming Logics for Denotations  of  Recursive  Objects}  (GR/R65190/01), \pounds 62,360) on program logics for Abadi and Cardelli's  object-calculus. The main results of this project include 
a correctness proof  of the Abadi-Leino logic \cite{AbadiLeino} using  denotational semantics  \cite{R6,RS10} that can explain the inherent limitations of the Abadi-Leino logic and
a new  logic for a  simple  imperative while-language where procedures  are first class data that can be stored away and updated at runtime,  but  not dynamically allocated   \cite{RS11}.  
 
 Reus has also obtained numerous  smaller grants from the DAAD, Nuffield Foundation,  EPSRC, and the London Mathematical Society, respectively, that funded   research visits and the organisation of events at Sussex.
  
\textbf{Usual stuff?   numbers of publications? other stuff?} The PI has published numerous refereed papers and articles in international conferences (LICS, POPL, CSL, ESOP, ICALP, FOSSACS, VMCAI) and journals (TCS, MSCS, JFP, LMCS) of high standing.     He also co-chaired and co-organised events  in both areas, notably   the now well established \emph{Domains}   and  \emph{Types}  workshops (in 1997, 2008 and in 1998, respectively) and co-edited journal volumes (TCS vol 264(2), MSCS vol 20(2))  and  an  LNCS volume (1657) with selected papers, respectively.
He has been the PC Chair for Domains IX (Sussex, 2008), and on the PC for Domains X (2011), Classical Logic and Computation (Brno, 2010). Reus has supervised two PhD students and externally examined two PhD students.  
%
  \section*{References}

 \newpage
 
\section*{Proposed Research}
up to 6 pages

\subsection*{Summary}
\label{sec:summary}

Dependently typed programming languages like Agda exploit a powerful type system to integrate programming and reasoning allowing us to freely move between prototypical programs and certifed deliverables. Dependently typed programming (DTP) is becoming increasingly popular and is being used for real world software like web servers and communication protocols (\emph{citation}). This raises an important issue: how to integrate effectful programming and dependent types in a way that supports the engineering of certified programs effectively? In particular we want to be able to program and reason about monotone resources like memory or threads. We are going to investigate how to use a semantic construction from category theory (preheaf models) to support the ubiquitous monotone resources building on previous work by  by Swierstra and Altenkirch (\emph{citation}) on functional specification of effects.


\subsection*{Introduction and Overview}
 Functional programming is  referentially transparent as side effects are prohibited or strictly controlled by a type system.  Modern  languages like Haskell,  Python, F\# and  object-oriented variants like OCaml and Scala  have massively contributed to the popularity of functional programming (citations needed). These languages are frequently used outside academia  (citation/examples needed, need some real life data).  (Patterns of functional Map-reduce???).
 
 One main feature of these languages is their strong type system which allows the programmer to  discover certain errors at compile time. If   the type system is rich enough such that it comprises dependent types, i.e. types that depend on program expressions, it can be used to encode program  properties (propositions-as-types). This gives rise to what is called \emph{Type theory}: a functional language  with a   type system strong enough  to express  (second-order) predicate logic. As a consequence, the verification of program properties can be done by the type checker which undoubtedly is of  huge benefit to program developers. Of course the programs will now also contain proof terms that the programmer, now also a verifier, needs to provide.
  
 Alas, functional programming is not sufficient for all applications. The so-called ``\emph{Awkward squad}'' (citation) is needed in reality, ie input/output, state, references, exceptions, concurrent threads. 
These are examples of  (impure) side effects, in functional programming modelled via monads (see Moggi/Wadler). 

 Our primary objective is to have a dependent type theory that ``includes''  such side effects (in a controlled way).
 We restrict to side effects caused by monotone resources like heap without deallocation or threads with possibility to  spawn new threads but not kill any threads.
 
 The major research question underlying this proposal is:
 \textbf{How can side effects on monotone resources be modelled inside type theory?}  
  There are two approaches to fix this problem:
 \begin{itemize}
 \item use Hoare Type  theory (references needed) where  effects and operations are \emph{added} to the type theory and \emph{axioms are postulated} all of which  need external justification. A Hoare triple type then is added that expresses the behaviour of the operations with effects. In this approach one does not build side effects into the type theory, one adds them  (disadvantages?)
 \item implement  effects and provide functional specifications for programs with effects along the lines of  Wouter Swierstra's work (citation). Here the effect is simulated using concrete data types.
However, in this approach  dependency on resources needs to be modelled explicitly. Moreover, type theory is total and so some ``smart constructors" (citation) are needed to encode everything nicely. The encodings are tedious and   not re-usable. Swierstra says: \emph{"Furthermore, the automatic weakening of references requires a decidable equality on our universe. This excludes references storing dependent types, such as dependent pairs or dependent functions. It would be interesting to investigate how to remove this restriction and better support the automatic weakening of functions that are polymorphic with respect to the shape of the heap, such as the inc function above."} (citation from his diss).
 \end{itemize}
 
 In this project we are looking to improve on the second approach.  We will try to develop  an elegant \emph{``embedded' solution}, an extension of core type theory with monotonic resources the semantics of which extends conservatively the standard semantics of type theory using an abstract data type that hides the implementation (ie the resource) but provides an interface to access the resource. We propose to do this first, by  means of a representative example, for the local state monad. We  will provide an adequate specification of the local state monad within dependent type theory and a presheaf model for it to ensure soundness.
 This kind of model is particularly apt as 
 (a) presheaves are adequate abstractions of the kind  of  dependency on a ``world'' which in our case is the resource (eg.\ the local state) in a monotone way,
 and (b) there already exist standard categorical models for (dependent) type theory where the category of types can be suitable instantiated by pre sheaf categories.
Categorical semantics delivers for free  a rich toolbox of concepts and theorems for  our models.  
 
In a dependent type they with monotone resources  reasoning about functional programs with resources (side effects) will be possible keeping all the useful and well-loved advantages of type theory (which are?). An additional benefit is that the development of such an extension via a ``presheaf application" can be regarded and maybe established as methodology for potentially other extensions of type theory. 
 
 \subsection*{Background}
 
 Presheaves,
 Type theory,  local state model, examples for resource usage?
 
 \subsection*{National Importance}
 \emph{Include a National Importance section within this document to justify why this proposal warrants support by the UK taxpayer. Describe how the potential benefits align with national priorities, how the research relates to EPSRC�s research areas and strategies, and how it complements other research activity in the field. It is anticipated that this section should not require more than one or two paragraphs for most proposals.}

 \emph{Applicants should include a brief description of the strategic fit of the proposal with respect to the portfolio we are trying to create as described in the "Our Portfolio" section of the EPSRC website.}
 
 This research concerns the improvement of language support for writing verified programs and thus fits perfectly in the EPSRC's "Verification and Correctness" theme. The development of languages and tools to support the production of correct software has numerous applications in safety critical systems, medical devices and even operating systems \emph{refer to Gerwin's verified OS? Msoft's verified device drivers?}

 
 The importance of this research area arises from the need to ensure that hardware and software is effective and reliable, but this is an increasing challenge as languages and architectures grow in complexity. Application areas such as security, transport, electronic design, safety critical systems and medical devices increasingly incorporate software. Many such applications require reassurances that software will not cause a system to behave in certain ways, or that a particular event will occur. Providing such guarantees employs many techniques included in the research area of Verification and Correctness. Indeed, in some sectors, the inclusion of mathematical proofs for behaviour is being incorporated into product development/validation standards.
The UK has recognised world leading groups and researchers involved in international collaborations with other leaders in the field, reflecting the high quality of this area within the ICT landscape. The need for the area itself, and the correct and verified systems it enables, is acknowledged by researchers from across the ICT portfolio. The application areas described above highlight the important contribution this area can make to the cross ICT priorities, of Many-Core Architectures and Concurrency in Embedded and Distributed Systems and Towards an Intelligent Information Infrastructure. Researchers working in this area are encouraged to contribute to these priorities and connect to researchers working in other areas.
On the basis of this we will seek to grow investment in this area, relative to others in the portfolio, in order to ensure there is sufficient research capability which can contribute to these priorities and engage with other research areas.

We fit  best into their 
Your grant "Reusability and Dependent Types" is in that theme as well and should  probably be mentioned somewhere in this proposal.

 \emph{Relate the research proposed to the research area(s) and any strategic actions that are highlighted in them. Include information on where a successful research programme would position the UK in the global research landscape - who are the international leaders in the area?}
 
 International leaders UK anyway? Strathclyde, Oxford, Nottingham?
 
 \emph{long-term research which delivers outcomes typically over a 10 � 50 year timescale}
 
 What to say here?
 
\emph{ Where do we meet  national strategic needs by establishing or maintaining a unique world leading research activity ( addressing key UK societal challenges, contributes to current or future UK economic success and/or enables future development of key emerging industry(s)) ?}

 Whatever we write won't it sound far fetched? Only needs 2 paragraphs luckily
 
 
 \subsection*{Academic Impact}
 
\emph{ Describe how the research will benefit other researchers in the field and in related disciplines, both within the UK and elsewhere.}

clear for the academic community but further afield?

\emph{What will be done to ensure that they can benefit?
Explain any collaboration with other researchers and their role in the project.}

\emph{ For each Visiting Researcher, set out why they are the most appropriate person, and what they will contribute to the project.}

visiting researchers?  Who shall we nominate?

Wouter Swierstra, Greg Morrissett?   Thomas Streicher, Jeremy Gibbons?

We should then say which work packages they will contribute for. Project partners should send letter of support as far as I understand.

Wed on;t have any industrial partners/visitors. is that a problem?

 \subsection*{Research Hypothesis and Objectives}
 
 \emph{Set out the research idea or hypothesis.
Explain why the proposed project is of sufficient timeliness and novelty to warrant consideration for funding.
Identify the overall aims of the project and the individual measurable objectives against which you would wish the outcome of the work to be assessed.}


Our hypothesis: we can safely embed monotone resources in type theory and this leads to   elegant programs that make side effects amenable to the functional programmer.

Objectives
\begin{itemize}
\item develop the syntactic language for the extension of dependent type theory with resources
\item develop the pre sheaf models for soundness
\item implement (what?) in Agda
\item prove the usability doing examples like local state (and what else?)
\end{itemize}
  \subsection*{Programme and Methodology}

 \emph{Detail the methodology to be used in pursuit of the research and justify this choice.
Describe the programme of work, indicating the research to be undertaken and the milestones that can be used to measure its progress. The detail should be sufficient to indicate the programme of work for each member of the research team. Explain how the project will be managed.}
 

We detail the work packages. Before we need to point out that we use the local state monad as guiding example, that we use pre sheaf semantics as we know it can deal with such monotone resource already. 
\spacing{0.85}
 \bibliographystyle{alpha}
 
 
\newpage
\section*{Work Plan}
one page, they like GANTT or PERT charts.

 \begin{enumerate}
 \item Define Presheaf semantics for core Type Theory
 \item Model the monotone resources in the above
 \item Agda implementation
 \item Examples (like IORef)
 \item Develop generic interface for Type Theory with monotone resources by reflection of semantics 
 \item Investigate combination of various resources and resource models
 \end{enumerate}
 
\subsection*{Task list}

\begin{enumerate}
 \item 
 \item 
 \item 
 \end{enumerate}
 

\newpage
\section{Justification of Resources}

up to 2 pages
\subsection{Directly Incurred Costs}
\paragraph*{Staff} 

 We request two Research Assistants (scale 6 or 7(?)) to help with ? \textbf{need to make twos fit with the workplan.}
 Ideally, the first RA would have some experience in one or several of the following: (dependent) type theory, effect systems, category theory, sheaf semantics.  The second RA would ideally have a good grasp of functional  programming, ideally with Agda and or Coq and good knowledge about side effects and/or category theory. 
 
 
\paragraph*{Travel and Subsistence} To present (partial) results and keep contact to other researchers in the same or related area support for attendance of  5  international workshops or conferences for both PIs and RFs    e.g.\  ETAPS, POPL, LICS, CSL, ICALP,  Types Workshops as well as national meetings such as BTCS and ``Fun in the Afternoon''.  The list of conferences is provisional, precise plans cannot be made until the details of the conferences' locations, dates and programmes are known. Note that we calculate an average cost of 900\pounds but some trips inside Europe will be cheaper but this is compensated by trips to Asia or the US which will be more expensive.  (4 times 5 times \pounds1,000    = \pounds 20,000 ).

To establish communication between the two sites involved  trips between Sussex and Nottingham are requested as follows:  4 times 4 one day day visits  by each member (\pounds 70 times 16 = \pounds 1,120)  and  3 times 4 two day visits  (6 times   \pounds  150 =  900\pounds).
  
Also included are short trips within  the UK for an exchange with researchers working in the same area, e.g.\ Birmingham (Levy), London (?) (10 times 80\pounds plus 10 times 18\pounds = 980\pounds).

For each Visiting Researcher we plan two one week (or one two week) stays at one of the  project sites. Only travel and subsistence need to be  budgeted for each of their  stays (n times 600\pounds = ?).

 
In order to maximise impact and for networking purposes we plan to organise a small specialist workshop in the topic. For this we request \pounds2000 to to pay for travel and subsistence of speakers and \pounds 500 for coffee and meals.

\paragraph*{Consumables} 
The two RAs will require a PC. The University of Nottingham and Sussex, resp., do not provide machines for RAs so two such PCs (\pounds 1000 each) are requested.
 
 \subsection{Directly Allocated Costs}
 Investigators will work about 7 to 8 hours a week?

?




\subsection{Exceptions}
  
 
 \newpage
 
 \section{Pathways to Impact}
 up to 2 pages
 
 \emph{Use this annex to the proposal to describe activities that can be undertaken during the project to accelerate the route to the identified benefits being realised; shortening the time between discovery and use of knowledge. Also identify the additional resources needed to undertake these activities. }
 
 \emph{In summary, the document should describe the kinds of impact envisaged, how the proposed research project will be managed to engage users and beneficiaries and increase the likelihood of impacts, including (wherever appropriate):
    Methods for communications and engagement\\
    Collaboration and exploitation in the most effective and appropriate manner\\
    The project team�s track record in this area\\
    The resources required for these activities.\\
     Please ensure these are also captured in the financial summary and the Justification of Resources.
}

\end{document}










 